\documentclass[11pt,a4paper,DIV=10,parskip=half,BCOR=0mm]{scrartcl}
\usepackage[utf8]{inputenc}
\usepackage[T1]{fontenc}
\usepackage[ngerman]{babel}
\usepackage{scrpage2} % pagestyle
\usepackage{geometry} % Seitenränder
\usepackage{amsmath}
\usepackage{amsfonts}
\usepackage{amssymb}
\usepackage{amsthm}
\usepackage{multicol}
\usepackage{graphicx}
\usepackage{float}
\usepackage{marginnote} % Randnotizen
\usepackage{tabularx} % Tabelle mit spaltentyp 'X' für feste breite

% Seitenränder einstellen: marginparwidth wichtig für Randnotizen!
\geometry{a4paper, top=20mm, left=35mm, right=50mm, bottom=30mm,
	headsep=10mm, footskip=12mm, marginparwidth=45mm, marginparsep=3mm}
	
% Fußzeile
\pagestyle{scrheadings}
\ifoot{ \hrule Erstellt am: \today von C. van Heteren-Frese}
\begin{document}
\pagenumbering{gobble}
%
% === HEADER ====
\newcolumntype{T}{>{\Large\centering\arraybackslash}X}
\setlength{\tabcolsep}{3mm} % kein Innenrand bei Spaleten
\noindent
\begin{tabularx}{\textwidth}{|l|T|l|}
\hline
Datum: & \rule{0pt}{5mm} \textbf{\textsf{Thema: Interessengruppen des Smart Grid}} & Blatt-Nr. \\
&&\\ 
\hline
\end{tabularx}
% === HEADER END ===
\rule{0pt}{3mm} \\
% === HEADER ====
\setlength{\tabcolsep}{0mm} % kein Innenrand bei Spaleten
\begin{tabularx}{\linewidth}{lXr}
{\Large\textsf{\textbf{Gruppe B:} Umweltorganisation \textit{GreenCity e.V.}}} & & \includegraphics[scale=0.04]{images/oeko}\\
\hline
\end{tabularx}
% === HEADER END ===
%
\subsection*{Hintergrund}
Du bist Mitglied der Umweltorganisation \textit{GreenCity e.V.} und vertrittst
mit deinen Kollegen die Auffassung, dass das Smart Grid notwendig für einen sinnvollen Einsatz von erneuerbaren Energien ist. Somit leistet es einen wichtigen Beitrag zum Klima- und Umeweltschutz. Davon haben alle was!

Außerdem setzt Du dich mit deiner Organisation für den Datenschutz und
Verbraucherschutz ein. Ihr seid der Meinung, dass \marginnote{\footnotesize \textbf{personenbezogene Daten} können einer bestimmten Person zugeordnet werden}personenbezogene Daten nicht weitergegeben werden dürfen.
\subsection*{Infos}
\begin{enumerate}
	\item[•]Der Nutzen des Smart Grids ist für die Gesellschaft sehr hoch.
	\item[•]Andererseits werden viele personenbezogene Daten erhoben, die leicht Missbraucht werden können. 
\end{enumerate}
\subsection*{Eure Position}
\begin{enumerate}
	\item[•]Ihr seid \textbf{für} das \textit{Smart Grid}
\end{enumerate}
\subsection*{Eure Gründe}
\begin{enumerate}
	\item[•]Ihr wisst, dass erneuerbare Energien sicher sind und zusammen mit einem Smart Grid genug Strom erzeugen.
	\item[•]Man kann auf Atomstrom verzichten. Der ist ohnehin zu gefährlich, wie man durch 
	Katastrophen (Tschernobyl, Fukushima) leider erfahren musste. 
	\item[•] Ihr seht die Sache aber auch skeptisch: \glqq Privatsphäre ist wichtig: Es gibt Sachen, die niemanden etwas angehen!\grqq
	
%Weil für das richtig
%Zusammenspiel der vielen einzelnen Komponenten eine Menge an
%Daten ausgetauscht werden müssen, seid ihr für hohe
%Sicherheitsstandards und einen ausgeprägten Datenschutz

\end{enumerate}
\end{document}