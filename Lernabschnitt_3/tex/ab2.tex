\documentclass[11pt,a4paper,DIV=10,BCOR=0mm]{scrartcl}
\usepackage[utf8]{inputenc}
\usepackage[T1]{fontenc}
\usepackage[ngerman]{babel}
\usepackage{scrpage2} % pagestyle
\usepackage{geometry} % Seitenränder
\usepackage{amsmath}
\usepackage{amsfonts}
\usepackage{amssymb}
\usepackage{amsthm}
\usepackage{multicol}
\usepackage{graphicx}
\usepackage{float}
\usepackage{marginnote} % Randnotizen
\usepackage{tabularx} % Tabelle mit spaltentyp 'X' für feste breite
\usepackage{fancybox}
\usepackage{array}	
 
% Seitenränder einstellen: marginparwidth wichtig für Randnotizen!
\geometry{a4paper, top=20mm, left=35mm, right=50mm, bottom=30mm,
	headsep=10mm, footskip=12mm, marginparwidth=45mm, marginparsep=3mm}
	
% Fußzeile
\pagestyle{scrheadings}
\ifoot{ \hrule Erstellt am: \today von C. van Heteren-Frese}
\begin{document}
\pagenumbering{gobble}
%
% === HEADER ====
\newcolumntype{T}{>{\Large\centering\arraybackslash}X}
\setlength{\tabcolsep}{3mm} % kein Innenrand bei Spaleten
\noindent
\begin{tabularx}{\textwidth}{|l|T|l|}
\hline
Datum: & \rule{0pt}{5mm} \textbf{\textsf{Thema: Interessengruppen des Smart Grid}} & Blatt-Nr. \\
&&\\ 
\hline
\end{tabularx}
% === HEADER END ===
\rule{0pt}{3mm} \\
% === HEADER ====
\setlength{\tabcolsep}{0mm} % kein Innenrand bei Spaleten
\begin{tabularx}{\linewidth}{lXr}
%\vspace{1mm}
{\Large\textsf{\textbf{AB 2 - Video "Datenschutz"}}} & &
\includegraphics[scale=0.25]{images/task}\\  % Logo rechtsbündig
\hline
\end{tabularx}
% === HEADER END ===
%
\subsection*{Aufgaben}
\begin{enumerate}
\item Schaue dir das Video an und mache dir zu folgendenPunkten Notizen:
\begin{enumerate}
\item Wer wird wo interviewt und worum geht es?
\item[] Wer:\vspace{1.2cm}
\item[] Wo:\vspace{1.2cm}
\item[] Was:\vspace{1.2cm}
\item Welcher Vorfall wird von dem Unternehmenssprecher angesprochen? Was ist da passiert?
\vspace{3cm}
\item Am Ende des Videos geht es kurz um die Angreifbarkeit des Smart Grids. Was
kann damit gemeint sein? Was könnte man einschleusen, um Schaden
anzurichten? \vspace{2cm}
\end{enumerate}
\item Was könnte man unter einem „gläsernen Menschen“ verstehen?
\end{enumerate}

\end{document}